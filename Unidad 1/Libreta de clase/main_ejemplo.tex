%%%%%%%%%%%%%%%%%%%%%%%%%%%%% Define Article %%%%%%%%%%%%%%%%%%%%%%%%%%%%%%%%%%
\documentclass{article}
%%%%%%%%%%%%%%%%%%%%%%%%%%%%%%%%%%%%%%%%%%%%%%%%%%%%%%%%%%%%%%%%%%%%%%%%%%%%%%%

%%%%%%%%%%%%%%%%%%%%%%%%%%%%% Using Packages %%%%%%%%%%%%%%%%%%%%%%%%%%%%%%%%%%
\usepackage{geometry}
\usepackage{graphicx}
\usepackage{amssymb}
\usepackage{amsmath}
\usepackage{amsthm}
\usepackage{empheq}
\usepackage{mdframed}
\usepackage{booktabs}
\usepackage{lipsum}
\usepackage{graphicx}
\usepackage{color}
\usepackage{psfrag}
\usepackage{pgfplots}
\usepackage{bm}
%%%%%%%%%%%%%%%%%%%%%%%%%%%%%%%%%%%%%%%%%%%%%%%%%%%%%%%%%%%%%%%%%%%%%%%%%%%%%%%

% Other Settings

%%%%%%%%%%%%%%%%%%%%%%%%%% Page Setting %%%%%%%%%%%%%%%%%%%%%%%%%%%%%%%%%%%%%%%
\geometry{a4paper}

%%%%%%%%%%%%%%%%%%%%%%%%%% Define some useful colors %%%%%%%%%%%%%%%%%%%%%%%%%%
\definecolor{ocre}{RGB}{243,102,25}
\definecolor{mygray}{RGB}{243,243,244}
\definecolor{deepGreen}{RGB}{26,111,0}
\definecolor{shallowGreen}{RGB}{235,255,255}
\definecolor{deepBlue}{RGB}{61,124,222}
\definecolor{shallowBlue}{RGB}{235,249,255}
%%%%%%%%%%%%%%%%%%%%%%%%%%%%%%%%%%%%%%%%%%%%%%%%%%%%%%%%%%%%%%%%%%%%%%%%%%%%%%%

%%%%%%%%%%%%%%%%%%%%%%%%%% Define an orangebox command %%%%%%%%%%%%%%%%%%%%%%%%
\newcommand\orangebox[1]{\fcolorbox{ocre}{mygray}{\hspace{1em}#1\hspace{1em}}}
%%%%%%%%%%%%%%%%%%%%%%%%%%%%%%%%%%%%%%%%%%%%%%%%%%%%%%%%%%%%%%%%%%%%%%%%%%%%%%%

%%%%%%%%%%%%%%%%%%%%%%%%%%%% English Environments %%%%%%%%%%%%%%%%%%%%%%%%%%%%%
\newtheoremstyle{mytheoremstyle}{3pt}{3pt}{\normalfont}{0cm}{\rmfamily\bfseries}{}{1em}{{\color{black}\thmname{#1}~\thmnumber{#2}}\thmnote{\,--\,#3}}
\newtheoremstyle{myproblemstyle}{3pt}{3pt}{\normalfont}{0cm}{\rmfamily\bfseries}{}{1em}{{\color{black}\thmname{#1}~\thmnumber{#2}}\thmnote{\,--\,#3}}
\theoremstyle{mytheoremstyle}
\newmdtheoremenv[linewidth=1pt,backgroundcolor=shallowGreen,linecolor=deepGreen,leftmargin=0pt,innerleftmargin=20pt,innerrightmargin=20pt,]{theorem}{Theorem}[section]
\theoremstyle{mytheoremstyle}
\newmdtheoremenv[linewidth=1pt,backgroundcolor=shallowBlue,linecolor=deepBlue,leftmargin=0pt,innerleftmargin=20pt,innerrightmargin=20pt,]{definition}{Definition}[section]
\theoremstyle{myproblemstyle}
\newmdtheoremenv[linecolor=black,leftmargin=0pt,innerleftmargin=10pt,innerrightmargin=10pt,]{problem}{Problem}[section]
%%%%%%%%%%%%%%%%%%%%%%%%%%%%%%%%%%%%%%%%%%%%%%%%%%%%%%%%%%%%%%%%%%%%%%%%%%%%%%%

%%%%%%%%%%%%%%%%%%%%%%%%%%%%%%% Plotting Settings %%%%%%%%%%%%%%%%%%%%%%%%%%%%%
\usepgfplotslibrary{colorbrewer}
\pgfplotsset{width=8cm,compat=1.9}
%%%%%%%%%%%%%%%%%%%%%%%%%%%%%%%%%%%%%%%%%%%%%%%%%%%%%%%%%%%%%%%%%%%%%%%%%%%%%%%

%%%%%%%%%%%%%%%%%%%%%%%%%%%%%%% Title & Author %%%%%%%%%%%%%%%%%%%%%%%%%%%%%%%%
\title{Apuntes de practicas}
\author{Rodrigo Miranda}
%%%%%%%%%%%%%%%%%%%%%%%%%%%%%%%%%%%%%%%%%%%%%%%%%%%%%%%%%%%%%%%%%%%%%%%%%%%%%%%

\begin{document}
    \maketitle
    Ejemplos de practica AMN941  - Usando Latex en VSCode
    \section*{Punto Fijo}
    \subsection*{Ejemplo 1}Use el método de punto para encontrar una raíz real de la ecuación
    \[
        f(x)= x^{3}+2x^{2}+10x-20 =0
        \]
    Empleando como valor inicial $x0=1$. Emplee $15$ decimales y una precisión de $10^{-5}$.


    \textbf{Solucion:} 

    \noindent Para punto fijo, debemos obtener una ecuacion $g(x)=x$ 


    \textbf{Opcion 1:} $x= x^{3}+2x^{2}+11x-20$
    
    \noindent \\ Verificamos que la ecuacion converga en el punto dado, para ello derivamos la ecuacion $g(x)$ y evaluamos en el punto, de manera que:

    $g'(x)=3x^{2}+4x+11 \longrightarrow g'(1)=18$

    \noindent \\ Esto lo podemos comprobar rapidamente tambien en matlab, de la siguiente manera
    \begin{figure}[ht]
        \includegraphics[scale=1]{img/ejemplo_clasePunto.png}
        \caption[Derivada y evaluacion g'(x)]{Matlab 1}
    \end{figure}
    \\Por lo tanto, incumplimos el teorema de Banach quien expresa que $|g'(x)|<1$ \pagebreak
    
    \textbf{Opcion 2: }$\frac{20}{x^2+2x+10}$
    \noindent \\Probamos esta ecuacion y derivada evaluada en matlab
    \begin{figure}[ht]
        \includegraphics*[scale=0.9]{img/ejemplo_clasePunto_2.png}
        \caption[Derivada y evaluacion g'(x)]{Matlab 2}
    \end{figure}
    \\Como podemos observar, el resultado de la evaluacion es menor a 1, por lo tanto si converge.

    \noindent Probaremos esta ecuacion con el metodo de punto fijo en Matlab.
    \begin{figure}[ht]
        \includegraphics[scale=0.8]{img/ejemplo_clasePunto_3.png}
        \caption[Metodo Punto Fijo]{Matlab 3}
    \end{figure}
    \\El valor aproximado de $x=1.368810031675092$


    \section*{Newton}
    \subsection*{Ejemplo 1}
    Use el método de Newton - Raphson para encontrar una solución exacta con una exactitud de $10^{-12}$ para la siguiente ecuación. Emplee 15 decimales: 
    \[
        \ln(x-1)+cos(x-1)=0 ; [1.3,2]
    \]
    \begin{figure}[ht]
        \includegraphics*[scale=0.9]{img/ejemplo3.png}
    \end{figure}
    
    \textbf{Nota:} Cuando estemos trabajando ejercicios, es importante analizar la grafica. Dado que no podemo modificar la ecuacion como en el metodo de punto fijo, debemos elegir correctamente nuestro valor inicia,
    para esto, debemos alegarnos de: Puntos de infleccion, maximos y minimos relativos y extremos, en estas partes no nos funcionara este metodo.
    Ademas, si la primera derivada de la ecuacion es 0, no nos funcionara este metodo.
\end{document}