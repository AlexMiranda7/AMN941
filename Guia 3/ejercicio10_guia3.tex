%%%%%%%%%%%%%%%%%%%%%%%%%%%%% Define Article %%%%%%%%%%%%%%%%%%%%%%%%%%%%%%%%%%
\documentclass{article}
%%%%%%%%%%%%%%%%%%%%%%%%%%%%%%%%%%%%%%%%%%%%%%%%%%%%%%%%%%%%%%%%%%%%%%%%%%%%%%%

%%%%%%%%%%%%%%%%%%%%%%%%%%%%% Using Packages %%%%%%%%%%%%%%%%%%%%%%%%%%%%%%%%%%
\usepackage{geometry}
\usepackage{graphicx}
\usepackage{amssymb}
\usepackage{amsmath}
\usepackage{amsthm}
\usepackage{empheq}
\usepackage{mdframed}
\usepackage{booktabs}
\usepackage{lipsum}
\usepackage{graphicx}
\usepackage{color}
\usepackage{psfrag}
\usepackage{pgfplots}
\usepackage{bm}
%%%%%%%%%%%%%%%%%%%%%%%%%%%%%%%%%%%%%%%%%%%%%%%%%%%%%%%%%%%%%%%%%%%%%%%%%%%%%%%

% Other Settings

%%%%%%%%%%%%%%%%%%%%%%%%%% Page Setting %%%%%%%%%%%%%%%%%%%%%%%%%%%%%%%%%%%%%%%
\geometry{a4paper}

%%%%%%%%%%%%%%%%%%%%%%%%%% Define some useful colors %%%%%%%%%%%%%%%%%%%%%%%%%%
\definecolor{ocre}{RGB}{243,102,25}
\definecolor{mygray}{RGB}{243,243,244}
\definecolor{deepGreen}{RGB}{26,111,0}
\definecolor{shallowGreen}{RGB}{235,255,255}
\definecolor{deepBlue}{RGB}{61,124,222}
\definecolor{shallowBlue}{RGB}{235,249,255}
%%%%%%%%%%%%%%%%%%%%%%%%%%%%%%%%%%%%%%%%%%%%%%%%%%%%%%%%%%%%%%%%%%%%%%%%%%%%%%%

%%%%%%%%%%%%%%%%%%%%%%%%%% Define an orangebox command %%%%%%%%%%%%%%%%%%%%%%%%
\newcommand\orangebox[1]{\fcolorbox{ocre}{mygray}{\hspace{1em}#1\hspace{1em}}}
%%%%%%%%%%%%%%%%%%%%%%%%%%%%%%%%%%%%%%%%%%%%%%%%%%%%%%%%%%%%%%%%%%%%%%%%%%%%%%%

%%%%%%%%%%%%%%%%%%%%%%%%%%%% English Environments %%%%%%%%%%%%%%%%%%%%%%%%%%%%%
\newtheoremstyle{mytheoremstyle}{3pt}{3pt}{\normalfont}{0cm}{\rmfamily\bfseries}{}{1em}{{\color{black}\thmname{#1}~\thmnumber{#2}}\thmnote{\,--\,#3}}
\newtheoremstyle{myproblemstyle}{3pt}{3pt}{\normalfont}{0cm}{\rmfamily\bfseries}{}{1em}{{\color{black}\thmname{#1}~\thmnumber{#2}}\thmnote{\,--\,#3}}
\theoremstyle{mytheoremstyle}
\newmdtheoremenv[linewidth=1pt,backgroundcolor=shallowGreen,linecolor=deepGreen,leftmargin=0pt,innerleftmargin=20pt,innerrightmargin=20pt,]{theorem}{Theorem}[section]
\theoremstyle{mytheoremstyle}
\newmdtheoremenv[linewidth=1pt,backgroundcolor=shallowBlue,linecolor=deepBlue,leftmargin=0pt,innerleftmargin=20pt,innerrightmargin=20pt,]{definition}{Definition}[section]
\theoremstyle{myproblemstyle}
\newmdtheoremenv[linecolor=black,leftmargin=0pt,innerleftmargin=10pt,innerrightmargin=10pt,]{problem}{Problem}[section]
%%%%%%%%%%%%%%%%%%%%%%%%%%%%%%%%%%%%%%%%%%%%%%%%%%%%%%%%%%%%%%%%%%%%%%%%%%%%%%%

%%%%%%%%%%%%%%%%%%%%%%%%%%%%%%% Plotting Settings %%%%%%%%%%%%%%%%%%%%%%%%%%%%%
\usepgfplotslibrary{colorbrewer}
\pgfplotsset{width=8cm,compat=1.9}
%%%%%%%%%%%%%%%%%%%%%%%%%%%%%%%%%%%%%%%%%%%%%%%%%%%%%%%%%%%%%%%%%%%%%%%%%%%%%%%

%%%%%%%%%%%%%%%%%%%%%%%%%%%%%%% Title & Author %%%%%%%%%%%%%%%%%%%%%%%%%%%%%%%%
\title{Title}
\author{Haoyun Qin}
%%%%%%%%%%%%%%%%%%%%%%%%%%%%%%%%%%%%%%%%%%%%%%%%%%%%%%%%%%%%%%%%%%%%%%%%%%%%%%%

\begin{document}
\section*{Ejercicio 10}
10. Una ventana circular de observación en un buque para la investigación marina tiene un radio de 1.5 pie, 
y el centro de la ventana se encuentra a 20 pies de distancia del nivel del agua. Determine la fuerza del 
fluido sobre la ventana, considere el peso específico del agua de mar igual a 64 lb/pie3, empleando la 
regla compuesta de Newton-Cotes con n=6. Además, obtenga el valor exacto y el error de aproximación. \\

\begin{figure}[ht]
    \includegraphics*[scale=0.55]{img/nc10_1.png}
\end{figure}
\textbf{Datos:}
\\Profundidad: $h(y)=20-y$
\\Longitud horizontal de la ventana: $L(y)=3x=3\sqrt[2]{1.5-y}$ \indent      $y\epsilon [-1.5,1.5]$
\\Densidad del agua de mar: $64lb/pie^3$
\\Formula cerrada de Newton-Cotes para n=6 $\int_{a}^{b} f(x) \,dx \cong \frac{h}{140}[41f(x0)+216f(x1)+27f(x2)+272f(x3)+27f(x4)+216f(x5)+41f(x6)]$


\noindent \\ La integral que buscamos quedaria de la siguiente manera:  $F= w\int_{a}^{b} h(y)L(y) \,dy$
\\$F= 64\int_{-1.5}^{1.5} (20-y)3(1.5-y^2) \,dy$
\\$F= 192\int_{-1.5}^{1.5} (20-y)(1.5-y^2) \,dy$ 
\\Ingresamos a matlab:
\\ a=-1.5;
\\ b=1.5;
\\$h=(b-a)/6$;
\\ syms x;
\\ $f=192*((20-x)*(1.5-x^2))$;
\\ $aprox6=double((h/140)*(41*subs(f,a)+216*subs(f,a+h)+27*subs(f,a+2*h)+272*subs(f,a+3*h)+27*subs(f,a+4*h)+216*subs(f,a+5*h)+41*subs(f,b)))$
\\
\\aprox6 =
\\
\\        8640
\\
\begin{figure}[ht]
    \includegraphics*[scale=0.55]{img/nc10_2.png}\\
    \includegraphics*[scale=0.8]{img/nc10_3.png}
\end{figure}
\end{document}